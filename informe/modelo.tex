\section{Modelo}

\subsection{Empleados que atendieron clientes mayores de edad}

\begin{listing}
\begin{minted}[frame=single,
               framesep=3mm,
               linenos=true,
               xleftmargin=21pt,
               tabsize=4]{js}
{
  "_id" : ObjectId("5622bf41228da935bd5e0a6a"),
  "nroLegajo" : 234,
  "nombre" : "Pepito Suarez",
  "clientes_mayores" : [
    {
      "_id" : ObjectId("5622c1d3228da935bd5e0a6b"),
      "fecha" : ISODate("2015-10-01T00:00:00Z")
    }
  ],
  "clientes_menores" : [ ],
  "sectores" : [
    {
      "sector" : "Comestibles",
      "tarea" : "Gerente"
    },
    {
      "sector" : "Indumentaria deportiva",
      "tarea" : "Supervisor"
    }
  ]
}
\end{minted}
\caption{Ejemplo Empleado}
\label{json-example}
\end{listing}

\textbf{Consulta: } db.empleados.find(\{``clientes_mayores": \{\$exists: true, \$not: \{\$size: 0\}\}\})

\vspace{3em}

\subsection{Artículos más vendidos}
\begin{listing}
\begin{minted}[frame=single,
               framesep=3mm,
               linenos=true,
               xleftmargin=21pt,
               tabsize=4]{js}
{
  "_id" : ObjectId("5622d234228da935bd5e0a6f"),
  "codBarras" : 342391,
  "nombre" : "Pro Speed Z-7",
  "sector" : "Calzado",
  "cant_unidades_vendidas" : 0
}
\end{minted}
\caption{Ejemplo Artículo}
\label{json-example}
\end{listing}

\textbf{Consulta: }
\begin{enumerate}
  \item max_cant_unidades_vendidas = (db.articulos.aggregate([\{\$group : \{_id: null, max : \{\$max : ``\$cant_unidades_vendidas"\}\}\}])).next().max
  \item db.articulos.find(``cant_unidades_vendidas" : max_cant_unidades_vendidas)
\end{enumerate}

\vspace{3em}

\subsection{Sectores donde trabajan exactamente 3 empleados}
\begin{listing}
\begin{minted}[frame=single,
               framesep=3mm,
               linenos=true,
               xleftmargin=21pt,
               tabsize=4]{js}
{
  "_id" : ObjectId("5622df91228da935bd5e0a75"),
  "codSector" : "Comestibles",
  "empleados" : {
    "total" : 1,
    "lista" : [ ObjectId("5622bf41228da935bd5e0a6a") ]
  }
}
\end{minted}
\caption{Ejemplo Sector}
\label{json-example}
\end{listing}

\textbf{Consulta: } db.sectores.find(\{``empleados.lista": \{\$size: 3\}\})

\vspace{3em}

\subsection{Empleado que trabaja en más sectores}

\

\textbf{Consulta: }

\begin{enumerate}
  \item var max = db.empleados.aggregate([\{\$group: \{\_id:null, max: \{\$max: \{\$size: ``\$sectores"\}\}\}\}]).next().max
  \item db.empleados.find(\{sectores: \{\$size: max\}\})
\end{enumerate}

\vspace{3em}

\subsection{Ranking de los clientes con mayor cantidad de compras (total de unidades)}

\begin{listing}
\begin{minted}[frame=single,
               framesep=3mm,
               linenos=true,
               xleftmargin=21pt,
               tabsize=4]{js}
{
  "_id" : ObjectId("5622c1d3228da935bd5e0a6b"),
  "dni" : 28012849,
  "nombre" : "Julio Jericho",
  "edad" : 23,
  "articulos" : {
    "total" : 4,
    "lista" : [ {"id": ObjectId("ff20ef41228da935bd5583bd"), "cantidad": 3},
                {"id": ObjectId("4729bce098bbddee98100acc"), "cantidad": 1}
              ]
  }
}

\end{minted}
\caption{Ejemplo Cliente}
\label{json-example}
\end{listing}

Cuando el cliente compra una cantidad determinada de un producto, se agrega una nueva entrada al
final del arreglo \emph{lista}, sumando al valor de total la cantidad de unidades compradas

\vspace{3em}

\subsection{Cantidad de compras realizadas por clientes de la misma edad}

\textbf{Consulta: }

db.clientes.aggregate([\{\$project: \{``art_total": ``\$articulos.total", ``edad": 1\}\}, \{\$group: \{_id: ``\$edad", total: \{ \$sum: ``\$art_total"\}\}\}])














