\section{Modelo}

Como regla general, el modelo está pensado en función de las consultas que se desea responder.
Para ello, utilizamos la desnormalización y la redundancia de datos como mecanismos válidos para
poder resolver las consultas rápida y eficientemente.

Esto produce como consecuencia la posibilidad de producir inconsistencias en la base de datos.
Por lo tanto, es responsabilidad del programador realizar los chequeos correspondientes a la hora
de insertar, modificar o eliminar datos.

\subsection{Empleados que atendieron clientes mayores de edad}

Para poder satisfacer los requerimientos de la query, mantenemos en la colección \emph{Empleados}
una lista de clientes, que contiene los identificadores de cada uno. Para esto utilizamos 2 arreglos,
diferenciando entre clientes mayores y menores de edad.

\begin{listing}
\begin{minted}[frame=single,
               framesep=3mm,
               linenos=true,
               xleftmargin=21pt,
               tabsize=4]{js}
{
  "_id" : ObjectId("5622bf41228da935bd5e0a6a"),
  "nroLegajo" : 234,
  "nombre" : "Pepito Suarez",
  "clientes_mayores" : [
    {
      "_id" : ObjectId("5622c1d3228da935bd5e0a6b"),
      "fecha" : ISODate("2015-10-01T00:00:00Z")
    }
  ],
  "clientes_menores" : [ ],
  "sectores" : [
    {
      "sector" : "Comestibles",
      "tarea" : "Gerente"
    },
    {
      "sector" : "Indumentaria deportiva",
      "tarea" : "Supervisor"
    }
  ]
}
\end{minted}
\caption{Ejemplo Empleado}
\label{json-example}
\end{listing}

\textbf{Consulta: } db.empleados.find(\{``clientes_mayores": \{\$exists: true, \$not: \{\$size: 0\}\}\})

\vspace{3em}

\subsection{Artículos más vendidos}

Agregamos en un atributo \emph{cant\_unidades\_vendidas} que nos permite encontrar el máximo
muy eficientemente.
No mantenemos información acerca de los clientes que compraron ese producto, ya que esta
última colección es la encargada de llevar ese registro.

\begin{listing}
\begin{minted}[frame=single,
               framesep=3mm,
               linenos=true,
               xleftmargin=21pt,
               tabsize=4]{js}
{
  "_id" : ObjectId("5622d234228da935bd5e0a6f"),
  "codBarras" : 342391,
  "nombre" : "Pro Speed Z-7",
  "sector" : "Calzado",
  "cant_unidades_vendidas" : 0
}
\end{minted}
\caption{Ejemplo Artículo}
\label{json-example}
\end{listing}

\textbf{Consulta: }

En primer lugar, realizamos una agregación con un único grupo: \emph{\_id: null}, mediante la cual
encontramos el máximo entre todas las cantidades vendidas.
Luego realizamos una nueva query, fitrando por igualdad sobre dicha máxima cantidad obtenida.

\begin{enumerate}
  \item max_cant_unidades_vendidas = (db.articulos.aggregate([\{\$group : \{_id: null, max : \{\$max : ``\$cant_unidades_vendidas"\}\}\}])).next().max
  \item db.articulos.find(\{``cant_unidades_vendidas" : max_cant_unidades_vendidas\})
\end{enumerate}

\subsection{Sectores donde trabajan exactamente 3 empleados}

La colección \emph{Sectores} mantiene una lista con los ids de los empleados que trabajan en él.

\begin{listing}
\begin{minted}[frame=single,
               framesep=3mm,
               linenos=true,
               xleftmargin=21pt,
               tabsize=4]{js}
{
  "_id" : ObjectId("5622df91228da935bd5e0a75"),
  "codSector" : "Comestibles",
  "empleados" : {
    "lista" : [ ObjectId("5622bf41228da935bd5e0a6a") ]
  }
}
\end{minted}
\caption{Ejemplo Sector}
\label{json-example}
\end{listing}


\textbf{Consulta: } db.sectores.find(\{``empleados.lista": \{\$size: 3\}\})

\vspace{3em}

\subsection{Empleado que trabaja en más sectores}

\

\textbf{Consulta: }

La consulta es análoga a ``artículos más vendidos". En primera instancia encontramos el máximo, y luego
seleccionamos aquellos documentos con valor igual a dicho máximo.

\begin{enumerate}
  \item var max = db.empleados.aggregate([\{\$group: \{\_id:null, max: \{\$max: \{\$size: ``\$sectores"\}\}\}\}]).next().max
  \item db.empleados.find(\{sectores: \{\$size: max\}\})
\end{enumerate}

\vspace{3em}

\subsection{Ranking de los clientes con mayor cantidad de compras (total de unidades)}

El cliente mantiene una lista con los artículos que compra.
Cuando el cliente compra una cantidad determinada de un producto, se agrega una nueva entrada al
final de \emph{lista}, sumando al valor de total la cantidad de unidades compradas. Este campo
admite repetidos sobre \emph{id}, lo cual agrega eficiencia a la hora de insertar una nueva compra.

\vspace{3em}

\begin{listing}
\begin{minted}[frame=single,
               framesep=3mm,
               linenos=true,
               xleftmargin=21pt,
               tabsize=4]{js}
{
  "_id" : ObjectId("5622c1d3228da935bd5e0a6b"),
  "dni" : 28012849,
  "nombre" : "Julio Jericho",
  "edad" : 23,
  "articulos" : {
    "total" : 4,
    "lista" : [ {"id": ObjectId("ff20ef41228da935bd5583bd"), "cantidad": 3},
                {"id": ObjectId("4729bce098bbddee98100acc"), "cantidad": 1}
              ]
  }
}

\end{minted}
\caption{Ejemplo Cliente}
\label{json-example}
\end{listing}

\textbf{Consulta: }

El ranking está determinado por un ordenamiento descendente sobre el total de compras
realizadas por cada cliente.

~

db.clientes.aggregate([\{\$sort : \{``articulos.total" : -1\} \} ])

\subsection{Cantidad de compras realizadas por clientes de la misma edad}

\textbf{Consulta: }

Agrupamos los documentos por edad proyectando el total de artículos comprados.
Finalmente realizamos una agregación sobre cada grupo, sumando $total$.

~

db.clientes.aggregate([\{\$project: \{``art_total": ``\$articulos.total", ``edad": 1\}\}, \{\$group: \{_id: ``\$edad", total: \{ \$sum: ``\$art_total"\}\}\}])














