\section{MapReduce}

\subsection{Cantidad de disposiciones de tipo resoluciones realizadas en Abril del 2013}

Por cada disposición, si cumple con las restricciones emitimos como clave: ``Resoluciones", y
1 como valor. El reduce consiste simplemente en la suma de todos los values.

\begin{lstlisting}
var m = function(){
  var date = convertDate(this.FechaDisposicion);
  if(date){
    var month = date.getMonth();
    if (month == 3 && this.Tipo == "Resoluciones"){
      emit(this.Tipo, 1);
    }
  }
}

var r = function(key,values) {
    return Array.sum(values)
}

printjson(db.disposiciones.mapReduce(m,r,{out: {inline: 1},
          scope: {convertDate: convertDate}}));
\end{lstlisting}

\vspace{4em}

\subsection{Cantidad de disposiciones por tipo definido}

Este caso consiste en un agrupamiento por ``Tipo" con la suma de la cantidad de documentos
como valor. Por lo tanto, la clave en este caso es el tipo de la disposición. Al igual que
en el caso anterior, se emite 1 como valor por cada documento y al reducir se suman todos
los valores.

\begin{lstlisting}
var m = function(){
  if (this.Tipo != ""){
    emit(this["Tipo"], 1);
  }
}
var r = function(key, values) {
  return Array.sum(values);
}

printjson(db.disposiciones.mapReduce(m,r,{out: {inline: 1}}));
\end{lstlisting}

\subsection{Fecha más citada para todos los informes}

Utilizamos la función convertDate para transformar las distintas fechas a un formato
común. Las fechas se presentan como day-month-year, day/month/year o en formato iso:
2015-11-21T01:51:17Z.

Las claves son las fechas normalizadas, y emitimos 1 como valor tanto por la \emph{FechaBOJA}
como por la \emph{FechaDisposicion}.

\begin{lstlisting}

var m = function(){
  var fecha_boja = convertDate(this.FechaBOJA);
  var fecha_disp = convertDate(this.FechaDisposicion);
  if (fecha_boja){
    emit(fecha_boja, 1);
  }
  if(fecha_disp){
    emit(fecha_disp, 1);
  }
}
var r = function(key, values) {
  return Array.sum(values);
}


var res = db.disposiciones.mapReduce(m,r,{out: {inline: 1},
                                    scope: {convertDate: convertDate}})
printjson(res);

//Find max
var max = null;
for(var idx = 0; idx < res.results.length; ++idx){
  var item = res.results[idx];
  if(max){
    if (Math.max(max.value, item.value) != max.value) max = item;
  }
  else{
    max = item;
  }
}

printjson(max);

\end{lstlisting}

\subsection{Devolver la mayor cantidad de páginas utilizadas por cada tipo de disposición}

La consulta consiste en un agrupamiento sobre tipo de disposición
con agregación sobre el máximo de las páginas utilizadas.

Por lo tanto emitimos el tipo de documento como clave y la cantidad de páginas utilizadas
como valor. El reduce encuentra el máximo sobre la lista de valores.

\begin{lstlisting}
var m = function(){
  var pagina_inicial = this["PaginaInicial"],
    pagina_final = this["PaginaFinal"];
  emit(this["Tipo"], pagina_final - pagina_inicial)
}

var r = function(key, values){
  var max = 0;
  // Calculo del maximo
  for(var i = 0; i < values.length; ++i){
    if (values[i] > max){
      max = values[i];
    }
  }
  return max;
}

var res = db.disposiciones.mapReduce(m, r, {out: {inline: 1}})
printjson(res);
\end{lstlisting}

\subsection{Función para convertir las fechas}

\begin{lstlisting}
function convertDate(str){
  if(str == "") return null;
  if(/^\d{4}-\d{2}-\d{2}T\d{2}:\d{2}:\d{2}Z?$/.test(str)){
    return new Date(str.replace(/^00/, "20"));
  }
  else if(/^\d{2}[-\/]\d{2}[-\/]\d{4}$/.test(str)){
    var parts = str.split(/[-\/]/);
    var aux = parts[0];
    parts[0] = parts[2];
    parts[2] = aux;
    return new Date(parts.join("-"));
  }
  else{
    return null;
  }
}
\end{lstlisting}



