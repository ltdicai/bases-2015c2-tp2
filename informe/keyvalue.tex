\section{Key-Value como alternativa}

\subsection{General}
Elegimos ejemplificar los cambios que implicaría utilizar una base de datos Key-Value.
Este tipo de bases está muy relaciono con las bases de datos Document, consideradas como una versión más potente de las primeras.
Ambas mantiene una _clave_ relacionada con un único _valor_, y esta relación define el límite de atomicidad. Además,
si bien en ambos tipos de bases de datos pueden modelarse relaciones entre entidades, la coherencia y validez de estas relaciones
deben ser mantenidad por el programador desde el código de la aplicación.

La diferencia fundamental entre ambos tipos radica en que, en las bases Key-Values, el _valor_ solo se considera como una cadena de bits
sin semántica, por lo que cualquier información que allí se guarde resulta inútil a fines de consulta y procesamiento directo: en una
base Key-Value, la única manera de acceder a un dato es sabiendo bajo que clave se almacena. En cambio, las bases de datos de documentos sí otorgan
una semántica a los valores relacionados a una clave - la idea de un documento con campos definidos. Esto les permite proveer operaciones
de búsqueda y agregación más cercanas a aquellas encontradas en bases de datos relacionales.

\subsection{Punto 1}
Como en las bases Key-Value no hay acceso posible a los datos si no se
especifican las claves, a diferencia del modelo realizado en Document DB, necesitamos mantener
información explícita sobre las claves de las entidades que vamos a almacenar. Por esto, utilizaríamos un bucket Claves con

~

  claves[``empleados"] $\rightarrow$ {[nroLegajo]}

  claves[``productos"] $\rightarrow$ {[codigoDeBarra]}

  claves[``sectores"]  $\rightarrow$ {[codigoSector]}

  claves[``clientes"]  $\rightarrow$ {[dni]}

~

Las bases Key-Value tampoco nos proveen métodos para operar sobre los valores almacenados por clave,
puesto que la parte del valor es considera como una caja negra. Por esto, si se quiere tener
una estructura de clave-valor que no sea extremadamente rigida (como sería el caso
de guardar claves-valores del estilo [``empleadosQueAtendieronAdultos"] $\rightarrow$ \{[nroLegajo]\}),
necesitamos poder iterar sobre las claves y operar con los valores por fuera de la base.

Por ejemplo, podriamos definir la siguiente estructura (incompleta) con los datos necesarios para las consultas requeridas:
\begin{enumerate}
\item empleados[$<nroLegajo>$:``clientes"] $\rightarrow$ \{``adultos": int, ``total": int, ``clientesAtendidos": [dni]\}
\item articulos[$<codigoDeBarra>$] $\rightarrow$ \{``cantidadVendida": int\}
\item sectores[$<codigoSector>$] $\rightarrow$ \{``cantidadEmpleados":int, ``empleados":[nroLegajo]\}
\item clientes[$<dni>$] $\rightarrow$ \{cantidadCompras: int\}
\item empleados[$<nroLegajo>$:``sectores"] $\rightarrow$ \{``cantidadSectores": int, ``sectores":[codigoSector]\}
\item clientes[``compras":$<edad>$] $\rightarrow$ {``cantidadCompras": int}
\end{enumerate}

Entonces, para responder la consulta ``Los empleados que atendieron clientes mayores de edad", el programador debería:

\begin{itemize}
\item conseguir los nroLegajo de los empleados utilizando claves[``empleados"]
\item para cada nroLegajo, obtener el json almacenado en empleados[$<nroLegajo>$:``clientes"]
\item mantener una lista de los empleados (nroLegajo) con mayor valor en el campo ``adultos" del json obtenido.
\end{itemize}

\subsection{Punto 2}
Como comentamos anteriormente, las bases de datos Key-Value no proveen métodos para realizar operaciones sobre la información almacenada
en los valores. Por lo tanto, el esquema de Map Reduce debería ser implementado en una capa superior, ya sea directamente por el programador o por una
librería. Si no existe una librería que provea tal servicio, podría resultar mas conveniente proceder
tal como lo haríamos para las consultas del punto 1.


\subsection{Punto 3}
El concepto de Sharding puede aplicarse de la misma manera tanto a bases Key-Value como Document, y deben tenerse en cuentas
las mismas consideraciones sobre la distribución de las claves elegidas para realizar las particiones. Existe, sin embargo, una diferencia
importante a considerar:
dependiendo de como se manejen los atributos de una entidad en Key-Value (agregados en una clave$\rightarrow$valores o desagregados en varias clave:atributo$\rightarrow$valor), pueden
darse casos en que una misma entidad tenga sus atributos repartidos entre distintos shards.

