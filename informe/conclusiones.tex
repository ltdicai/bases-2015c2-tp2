\section{Conclusiones}

En el transcurso del desarrollo de este trabajo nos enfrentamos con problemas de diseño e implementación
típicos en el uso de bases NoSQL. Entre ellos:

\begin{itemize}
  \item Representar la información para prescindir del uso de join, tratando de mantener un balance entre
la consistencia de los datos y la atomicidad de las operaciones.
  \item  Extraer información
utilizando la técnica de MapReduce, en la cual el propio programador debe cumplir un rol que suele relegarse al motor de bases de dato relacionales
  \item Dividir de forma adecuada los datos a fin de lograr escalabilidad.
\end{itemize}

A partir de todo esto, llegamos a entender que las bases de datos NoSQL pueden ser muy útiles en situaciones particulares en las que, entre otras cosas:
\begin{itemize}
  \item Las consultas que se realizaran sobre los datos son prefijadas o predecibles
  \item No se requiera realizar consultas arbitrarias
  \item La consistencia de la informacion no es un factor relevante (aunque existen bases NoSQL que cumplen con ACID, no son la mayoría ni las más utilizadas)
  \item Se espera que el sistema deba procesar cantidades extremadamente grandes de información de manera eficaz
  \item Se espera que millones de usuarios utilicen el sistema de manera concurrente
\end{itemize}

Todas estas características no implican descartar el uso de bases de datos relaciones sino que, si dichas circunstancias están presente, permiten evaluar
la conveniencia de dejar de lado la complejidad en que incurren las bases relacionales que cumplen con las propiedades ACID por otras bases con una mecánica simplificada a costa de
no hacerlo.

En definitiva, las bases de datos NoSQL no vienen a reemplazar a las bases relacionales - las complementan, poniendo a disposición de los desarrolladores
más herramientas con las que encontrar soluciones eficaces a los problemas cada
vez más complejos con los que se enfrenta el mundo de la administración de información.
